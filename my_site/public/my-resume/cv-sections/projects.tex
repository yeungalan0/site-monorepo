%----------------------------------------------------------------------------------------
%       SPECTIAL COMMANDS
%----------------------------------------------------------------------------------------

\newcommand{\varLengthCells}[3]{\begin{tabular*}{\textwidth}{@{\extracolsep{\fill}} L{\textwidth - #3\textwidth} R{#3\textwidth}} #1 & #2 \end{tabular*}}

% Define an entry of cv information
% Usage: \cventry{<position>}{<title>}{<location>}{<date>}{<description>}{<width1>}{<width2>}
%% \newcommand*{\myProject}[7]{%
%%   \setlength\tabcolsep{0pt}
%%   \setlength{\extrarowheight}{0pt}
%%   \begin{tabular*}{\textwidth}{@{\extracolsep{\fill}} L{\textwidth - 4.5cm} R{4.5cm}}
%%     \varLengthCells{\entrytitlestyle{#2}}{\entrylocationstyle{#3}}{#6} \\
%%     \vspace{-1em}
%%     \varLengthCells{\entrypositionstyle{#1}}{\entrydatestyle{#4}}{#7} \\
%%     \multicolumn{2}{L{\textwidth}}{\descriptionstyle{#5}} \\
%%   \end{tabular*}
%% }

%----------------------------------------------------------------------------------------
%	SECTION TITLE
%----------------------------------------------------------------------------------------

\cvsection{Projects}

%----------------------------------------------------------------------------------------
%	SECTION CONTENT
%----------------------------------------------------------------------------------------

\begin{cventries}

    %------------------------------------------------

    % \vspace{-1em}
    % \cventry
    % {A simple search engine created to search through gigabytes of review data quickly and efficiently.} % Quick description
    % {Amazon Review Search Engine} % Title
    % {http://cs.coloradocollege.edu/~ayeung/reviewSI/SearchHome.html (link OOO))} % Location
    % {Nov. 2015 - Dec. 2015} % Date(s)
    % { % Description(s)
    % \begin{cvitems}
    % \item {Designed the web interface and document database to quickly search through Amazon review data given search criteria.}
    % \item {(Python, JavaScript, HTML, CGI, Redis, MongoDB)}
    % \end{cvitems}
    % }

    %------------------------------------------------

    % \vspace{-1em}
    % \cventry
    % {An online service providing professors with a list of students who need to be attended to.} % Description
    % {Professor's Queue Database} % Title
    % {} % Location
    % {Apr. 2014} % Date(s)
    % { % Description(s)
    % \begin{cvitems}
    % \item {Designed and implemented a web page and database with two friends to place students in different teacher queues so that students no longer needed to wait outside a professor's office to ask questions.}
    % \item {Personally designed the database structure and wrote CGI scripts to handle client input and maintain the server-side state. (Python, JavaScript, HTML)}
    % \end{cvitems}
    % }

    %------------------------------------------------

    % \vspace{-.25em}
    % \cventry
    % {Programmed the alpha-beta pruning algorithm for a bot that won the class competition} % Description
    % {Othello Bot} % Title
    % {https://github.com/yeungalan0/Othello} % Location
    % {Mar. 2015} % Date(s)
    % { % Description(s)
    % \begin{cvitems}
    % \item {Made a bot for a class project using object oriented programming, alpha beta pruning, and heuristics to play Othello strategically}
    % \item {The bot, won the class competition and is undefeated so far when played against other online bots (project used Java)}
    % \end{cvitems}
    % }

    %------------------------------------------------

    % \vspace{-1.5em}
    % \cventry
    % {Developed and trained an AI agent to perform intelligent actions in a Minecraft like environment} % Description
    % {Minecraft AI} % Title
    % {https://tinyurl.com/ayeung-minecraft} % Location
    % {2015-2016} % Date(s)
    % { % Description(s)
    % \begin{cvitems}
    % \item {Interfaced an existing Minecraft game environment with a neural network developed by Google DeepMind}
    % \item {Trained the agent to perform several tasks in the 3D environment and analyzed learning}
    % \item {Worked on developing a better algorithm through Transfer Learning}
    % \item {Published paper at IJCAI Deep Reinforcement Learning workshop based on our research (project used Python, Caffe, Torch, Lua, C++)}
    % \end{cvitems}
    % }

    %------------------------------------------------

    % \vspace{-2.25em}
    % \cventry
    % {Developed a program that queried several APIs and suggested books to users based on user criteria} % Description
    % {Wheel of Books} % Title
    % {} % Location
    % {Sep. 2015} % Date(s)
    % { % Description(s)
    % \begin{cvitems}
    % \item {Queried three different APIs (Google Books, ISBNdb, and the Fixer IO API)}
    % \item {Parsed XML and JSON responses from the APIs}
    % \item {Developed a simple UI to present the book information to the user (project used Python, XML, and JSON)}
    % \end{cvitems}
    % }

    %------------------------------------------------

    \vspace{-.25em}
    \cventry
    {My personal webpage for side project frontend hosting and blogging} % Description
    {My Site (work in progress)} % Title
    {\href{https://www.imakethings.dev/}{https://www.imakethings.dev/}} % Location
    {Feb. 2021 - Present} % Date(s)
    { % Description(s)
        \begin{cvitems}
            \item {Developed personal webpage primarily written in Typescript using React and NextJS frameworks}
            \item {Utilized Material UI for frontend components to add a more professional look}
            \item {Developed NextJS APIs for blog filtering and associated unit tests}
            \item {\scriptsize{\textbf{Tech used/learned:} ReactJS, NextJS framework, Typescript, Material UI, GitHub Actions for automated tests, Vercel deployment}}
        \end{cvitems}
    }

    %------------------------------------------------

    % \vspace{-.25em}
    % \cventry
    % {A webapp where people can pay \$1 to see how many other people paid \$1} % Description
    % {One Dollar Site} % Title
    % {\href{https://www.imakethings.dev/posts/one-dollar-site}{https://www.imakethings.dev/posts/one-dollar-site}} % Location
    % {Oct. 2020 - Jan. 2021} % Date(s)
    % { % Description(s)
    %     \begin{cvitems}
    %         \item {Developed NodeJS Lambda function backend to process and store incoming API requests}
    %         \item {Integrated backend code with PayPal SDK to verify payments}
    %         \item {Developed NextJS front end to take payments and display the current count to the user}
    %         \item {Deployed dev environment using automated pipelines, with front end accessible at \href{https://www.imakethings.dev/one-dollar-site}{https://www.imakethings.dev/one-dollar-site}}
    %         \item {\textbf{Tech used/learned:} ReactJS, NextJS framework, PayPal SDK, AWS Lambda/DynamoDB/API-Gateway, NodeJS, Serverless framework}
    %     \end{cvitems}
    % }

    %------------------------------------------------



    %% \vspace{-1em}
    %% \cventry
    %% {Refactored and updated the Time is Money Chrome extension} % Description
    %% {Time is Money Chrome Extension} % Title
    %% {\href{https://tinyurl.com/ayeung-timeismoney}{https://tinyurl.com/ayeung-timeismoney}} % Location
    %% {June - Sep. 2020} % Date(s)
    %% { % Description(s)
    %% \begin{cvitems}
    %% \item {Developed a popup actions menu to disable/enable the plugin (with icon changing depending on state)}
    %% \item {Developed a feature to only update the active tab on state change to decrease unneeded memory usage}
    %% \item {Refactored and updated the code for easier long term maintainability/understanding}
    %% \item {Made pull request to contribute back changes to main project}
    %% \item {\textbf{Tech used/learned:} JavaScript, Google Chrome Extension framework, HTML, CSS}
    %% \end{cvitems}
    %% }

    %------------------------------------------------

    \vspace{-1em}
    \cventry
    {Completed Advent of Code 2020 programming advent calender in Go} % Description
    {\href{https://adventofcode.com/}{Advent of Code 2020}} % Title
    {\href{https://www.imakethings.dev/posts/advent-of-code-2020}{https://www.imakethings.dev/posts/advent-of-code-2020}} % Location
    {Dec. 2020 - Jan. 2021} % Date(s)
    { % Description(s)
        \begin{cvitems}
            \item {Practiced key Test Driven Development concepts to guide my code creation and validate its quality}
            \item {Developed my knowledge of Go, while spreading holiday cheer at the same time}
            \item {Tested and improved my knowledge of programming concepts such as recursive algorithms, dynamic programming, linked lists, the Chinese Remainder theorem, and number theory}
            \item {\scriptsize{\textbf{Tech used/learned:} Golang, Gitpod}}
        \end{cvitems}
    }

    %------------------------------------------------

\end{cventries}
